%----------------------------------------------------------------------
\section[GNSS]{Global Navigation Satellite Systems}
%----------------------------------------------------------------------
 
%----------------------------------------------------------------------
\paragraph{Operation principles of a GNSS position fix}
%----------------------------------------------------------------------

The basic ideas behind the operation of a GNSS can be summarized as follows:
%
\begin{enumerate}

\item The satellites of a GNSS constellation each have a very precise on
board clock that is synchronized with all other clocks in the constellation.

\item The position vector of each satellite in the constellation is known (or
can be made known) to any user of the system at all times.

\item Each satellite broadcasts a time-stamped, identifiable signal in all
directions. Each signal encodes, among other pieces of information, the exact
time at which the signal was broadcast. 

\item Any number of users can receive these signals simultaneously: they are
all passive (i.e. receiving) users.

\item A user receiving a satellite signal can compare the broadcast time with
his/her own clock: the travel time for the signal $\Delta t = t_{RX} - t_{TX}$
is a direct measure of the distance (or range) $d$ between the user and the
emitting satellite, since the speed of light $c$  is a physical constant: $d =
c\Delta t$.

\item Assuming the instantaneous position of the satellites is known to the 
user, using 3 such distance measurements to 3 different satellites allows, in 
principle, to determine the 3 unknowns $x,y,z$ of the position of the user.

\item However, for this to work, the user would need to have his/her clock
perfectly synchronized with the constellation clocks. As this would be 
impractical, a 4th distance measurement with a 4th satellite is needed to 
determine the 4 unknowns of the problem: $x,y,z$ and $t$, the time of the user.

\end{enumerate}
%
Obtaining a position (and time) fix for a user requires therefore solving a 
system of 4 non-linear equations, one for each pseudo-range measurement:
%
\begin{align}
c^2(t_{RX} - t_{TX,1})^2 &= 
(x_{RX} - x_{TX,1})^2 + (y_{RX} - y_{TX,1})^2 + (z_{RX} - z_{TX,1})^2,
\\
c^2(t_{RX} - t_{TX,2})^2 &= 
(x_{RX} - x_{TX,2})^2 + (y_{RX} - y_{TX,2})^2 + (z_{RX} - z_{TX,2})^2,
\\
c^2(t_{RX} - t_{TX,3})^2 &= 
(x_{RX} - x_{TX,3})^2 + (y_{RX} - y_{TX,3})^2 + (z_{RX} - z_{TX,3})^2,
\\
c^2(t_{RX} - t_{TX,4})^2 &= 
(x_{RX} - x_{TX,4})^2 + (y_{RX} - y_{TX,4})^2 + (z_{RX} - z_{TX,4})^2,
\end{align}
%
where $x_{TX,i},y_{TX,i},z_{TX,i}$ is the position of the $i$-th transmitting
satellite, $t_{TX,i}$ the time at which the signal from the $i$-th satellite
was sent, and $x_{RX},y_{RX},z_{RX},t_{RX}$ are the unknowns (position and 
time of the receiving user).

%----------------------------------------------------------------------
\paragraph{Operation principles of a GNSS velocity fix}
%----------------------------------------------------------------------

Once the procedure to determine the position is clear, it would be possible to
compute the velocity of a user by observing how his/her position changes in
time. In other words, by taking the derivative of the position vector:
%
\begin{equation}
\bm v = \frac{\dd \bm r}{\dd t} \simeq \frac{\bm r(t_2)-\bm r(t_1)}{t_2-t_1}. 
\end{equation}
%

There is, however, another, more accurate way to determine the user's velocity:
using the Doppler shift of the signal frequency. This is done according to the
following basic ideas:
%
\begin{enumerate}

\item The exact frequency at which the satellite signals are emitted is known 
to the users of the system.

\item The velocity vector of each satellite in the constellation is known (or
can be made known) to any user of the system at all times.

\item Whenever the distance between a satellite and a user changes in time 
(either because the satellite is moving or the user is moving),
the perceived signal frequency by the user changes slightly: the frequency 
will increase if the two of them are coming closer together, and it will 
decrease if they are moving apart. This physical phenomenon is known as 
\emph{Doppler shift}.

\item Any user listening to the satellite signals can determine the Doppler 
shift of any of them, $\Delta f = f_{RX} - f_{TX}$.

\item As the user knows the velocity of each satellite, combining 3 such 
measurements with 3 different satellites 
it is possible to determine the user's velocity (3 unknowns: $v_x,v_y,v_z$).

\item However, the cheap user's clock may have a drift with time. This may
lead the user to wrong measurements of the Doppler shifts. Thus, a 4th 
measurement with a 4th satellite is needed to determine the drift rate of the 
user clock as an extra unknown.

\end{enumerate} 
%
Obtaining a velocity (and clock rate) fix for a user requires therefore
solving the following system of 4 non-linear equations:
%
\begin{align}
\dot{t}_{RX}f_{RX,1}-f_{TX,1} = \frac{f_{TX,1}}{c} (\bm v_{RX}-\bm v_{TX,1}) 
\cdot \frac{\bm r_{RX}-\bm r_{TX,1}}{\left|\bm r_{RX}-\bm r_{TX,1}\right|}
\\
\dot{t}_{RX}f_{RX,2}-f_{TX,2} = \frac{f_{TX,2}}{c} (\bm v_{RX}-\bm v_{TX,2}) 
\cdot \frac{\bm r_{RX}-\bm r_{TX,2}}{\left|\bm r_{RX}-\bm r_{TX,2}\right|}
\\
\dot{t}_{RX}f_{RX,3}-f_{TX,3} = \frac{f_{TX,3}}{c} (\bm v_{RX}-\bm v_{TX,3}) 
\cdot \frac{\bm r_{RX}-\bm r_{TX,3}}{\left|\bm r_{RX}-\bm r_{TX,3}\right|}
\\
\dot{t}_{RX}f_{RX,4}-f_{TX,4} = \frac{f_{TX,4}}{c} (\bm v_{RX}-\bm v_{TX,4}) 
\cdot \frac{\bm r_{RX}-\bm r_{TX,4}}{\left|\bm r_{RX}-\bm r_{TX,4}\right|}
\end{align}
%
where $\bm v_{TX,i}$ is the velocity of the $i$-th transmitting
satellite, $f_{TX,i}, f_{RX,i}$ the frequencies at which the signal from the 
$i$-th satellite was respectively broadcast and received,
and $\bm v_{RX}, \dot{t}_{RX}$ are the unknowns (position and 
clock drift rate of the receiving user). Note that the vector
$(\bm r_{RX}-\bm r_{TX,i})/\left|\bm r_{RX}-\bm r_{TX,i}\right|$ is a unit 
vector that points in the direction from the $i$-th satellite to the user.
