%----------------------------------------------------------------------
\section[Electric Propulsion]{Fundamentals of Electric Propulsion}
%----------------------------------------------------------------------
 
%----------------------------------------------------------------------
\paragraph{Key parameters of electric propulsion}
%----------------------------------------------------------------------

While chemical propulsion is \emph{energy limited}, as its performance depends
on the amount of energy per unit mass stored in the chemical bonds of the
propellant, electric propulsion is \emph{power limited}: the performance
depends on how  much electric power is available on board.

As with chemical propulsion, there are two basic parameters that characterize 
an electric thruster:
%
\begin{enumerate}

\item Specific impulse $I_{sp}$: this is a measure of the velocity at which
the propellant is ejected from the thruster. A higher specific impulse allows
fulfilling a propulsive mission (i.e. providing a given $\Delta V$ to the
spacecraft) at a much lower expense of propellant. Electric propulsion devices
have $I_{sp}^v$ in the order of $10$--$100$ km/s (about $1000$--$10000$ s if
$I_{sp}$is expressed in seconds), whereas the best chemical rockets can only provide about $5$ km/s (about $500$ s).

\item Thrust $F$: the force the propulsion system can generate to accelerate
the spacecraft. Contrary to chemical propulsion, where $F$ can be very large,
electric propulsion thrust levels are comparatively small: current thrusters
provide $< 1$ N of force (typically, about $50$--$200$ mN).  Thrust is equal
to the propellant mass flow rate $\dot m$ used in the thruster times the
exhaust velocity, i.e., its specific impulse in velocity units: 
%
\begin{equation}
F=\dot m I_{sp}^v.    
\end{equation}


\end{enumerate}
%
Electric propulsion provides tiny thrust levels, but it does so very 
efficiently, using very little propellant. This enables space missions that 
would be too ambitious to be carried out only with chemical propulsion, and to 
lower the cost of existing missions. 
Note, however, that electric propulsion cannot fully replace chemical 
propulsion: large thrust levels are required to take off from the surface of a 
planet, and for some quick propulsive maneuvers.

Apart from $I_{sp}$ and $F$, there is another important figure of merit of an 
electric thruster:
%
\begin{enumerate}[resume]

\item Thrust efficiency $\eta_T$: it is a measure of how the input power $P$
is  used for propulsion. The power contained in a jet of mass flow rate $\dot
m$  and exhaust velocity $I_{sp}^v$ is $\dot m (I_{sp}^v)^2/2$. The thrust
efficiency is defined as the ratio of the jet power over the input power:
%
\begin{equation}
\eta_T = \frac{\dot m (I_{sp}^v)^2}{2P} = \frac{ F I_{sp}^v}{2P} =  
\frac{ F^2}{2\dot m P} 
\end{equation}

\end{enumerate}