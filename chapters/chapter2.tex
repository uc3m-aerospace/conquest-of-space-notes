%----------------------------------------------------------------------
\section[Rockets]{Rocket motion}
%----------------------------------------------------------------------
 
%----------------------------------------------------------------------
\paragraph{Action-reaction principle: thrust force}
%----------------------------------------------------------------------

As the third law of Newton states, for every action there is an equal but
opposite action elsewhere. By ejecting propellant, a rocket exploits this law
to create a  thrust force on the vehicle equal to the momentum expelled from
the rocket per unit time:
%
\begin{equation}
F = \dot{m} c,
\end{equation}
%
where $\dot{m}=-\dd m/\dd t$ is the mass flow rate at which propellant 
is being ejected, and
$c$ the exhaust velocity of the propellant leaving the rocket.
The velocity $c$ is also known as \textbf{specific impulse}, $I_{sp}$. 
Due to historical reasons, the $I_{sp}$ is sometimes expressed in seconds
instead of in velocity units, dividing $c$ by $g_0=9.81$ m/s$^2$,
the gravity acceleration on the ground:
%
\begin{equation}
I_{sp}^v = c \mbox{ (in velocity units)}; \quad 
I_{sp} \,[\mbox{s}] = \frac{c \,[\mbox{m/s}]}{g_0 \,[\mbox{m/s}^2]}
\mbox{ (in seconds)}.
\end{equation}

%----------------------------------------------------------------------
\paragraph{Tsiolkovsky's rocket equation}
%----------------------------------------------------------------------

The equation of motion of a rocket system of mass $m$ and velocity $v$ with 
thrust $F=\dot{m}c$ is:
%
\begin{equation}
m\frac{\dd v}{\dd t} = \dot{m}c = -\frac{\dd m}{\dd t} c
\end{equation}
%
integrating this equation between two instants of time $t_1$ and $t_2$, 
between which the rocket velocity changes from $v_1$ to $v_2$, and
the rocket mass changes from $m_1$ to $m_2$,
requires some knowledge of differential equations. 
The result is:
%
\begin{equation}
\frac{\dd v}{c} = \frac{\dd m}{m} \Rightarrow \frac{v_2-v_1}{c}
=\ln\left(\frac{m_1}{m_2}\right).
\end{equation}
%
In this equation, ``$\ln$'' is the natural logarithm. We define the quantity
$\Delta v = v_2-v_1$, pronounced \emph{delta-vee}, as
the velocity increase obtained thanks to the rocket, which 
has used a mass of propellant equal to $m_2-m_1$.
Each space mission has a well-defined ``cost'' in terms of the 
necessary $\Delta v$ to accomplish it.

This equation can be inverted using the exponential function ``$\exp$'' into
%
\begin{equation}
\frac{m_2}{m_1} = \exp\left(-\frac{\Delta v}{c}\right) 
\equiv \exp\left(-\frac{\Delta v}{I_{sp} g_0}\right).
\end{equation}
%
This equation shows that unless the $\Delta v$ of the mission that we want to
carry out is smaller or comparable to the $I_{sp}g_0$ of our rocket technology,
the final mass $m_2$ will be tiny compared to the initial mass 
$m_1$ of the rocket. Typically, sending a small satellite into space requires
a large rocket launcher for this reason.

%----------------------------------------------------------------------
\paragraph{Rocket staging}
%----------------------------------------------------------------------

Tsiolkovsky's rocket equation includes in $m_2$ not only the payload $m_{pay}$
of the  rocket (e.g. the satellite we want to put into orbit), but also any
inert mass in the system such as structural mass of the rocket itself,
$m_{strut}$. Given the adverse scaling of the rocket size, anything that can
reduce inert mass is desirable.

Staging consists in releasing structural mass as soon as it is no longer 
needed (e.g. depleted propellant tanks, used-up rockets, etc). This way, the 
remaining of the trip is done with a lower inert mass, improving the overall
mass performance of the rocket system. In effect, this is the same as
stacking several rockets on top of each other, and igniting them in series.

To solve staging problems, it is useful to write down the initial mass of 
the $i$-th rocket stage as
%
\begin{equation}
m_{0,i} = m_{fuel,i} + m_{struct,i} + m_{pay,i},
\end{equation}
%
and note that the payload of the $i$-th stage is actually the next stage, 
$(i+1)$. The advantages of staging are great for systems with up to 3--4 
stages, but beyond that, the reduction of the initial system mass is very 
limited.

%----------------------------------------------------------------------
\paragraph{Impulsive maneuvers}
%----------------------------------------------------------------------

Rocket maneuvers allow to change the trajectory (and thus the orbit) of a
spacecraft. Chemical rockets provide large thrust levels for short periods of
time; hence, we can simplify the analysis of chemical rocket maneuvers by
imagining that the spacecraft changes instantaneously its velocity vector $\bm
v$ when the maneuver takes place. This approximation is known as the
\emph{impulsive maneuver model}. Using vector operations,  velocity changes
from $\bm v_1$ to $\bm v_2=\bm v_1 + \Delta \bm v$, while  the position vector
$\bm r$ remains unchanged during the instantaneous rocket firing.

The Hohmann transfer explained in the course is an example of application
of the impulsive maneuver model.