%----------------------------------------------------------------------
\section[Space Environment]{Space Environment}
%----------------------------------------------------------------------
 
%----------------------------------------------------------------------
\paragraph{Isothermal atmosphere model}
%----------------------------------------------------------------------

Earth's atmosphere is composed of many different layers. Temperature varies in
a non-trivial way, inverting its gradient several times. Pressure, however,
decreases monotonically as we ascend in the atmosphere. Modeling the 
atmospheric pressure and density in the upper layers of the atmosphere is 
important to determine air drag on low-Earth-orbit satellites.

We can pose a very simple model for pressure if we assume that temperature is
constant in the atmosphere $T=\const$. This is only an approximation of
course, and more advanced models exist that take temperature variation with
altitude into account. We call this model the ``isothermal atmosphere'' model.

Consider a small cylindrical column of air of height $\dd h$, and call $A$ the
area of its bottom and upper surfaces. If air density is $\rho$ at this height,
the weight of this column of air is simply $\rho g A \dd h$, where $g$ is 
the gravity acceleration.

Since this column is in equilibrium (i.e., it is not falling or rising), this
weight must be compensated by the pressure difference $\dd p$ between its
bottom  and upper surfaces. Then, we can write the following force balance
equation:
%
\begin{equation}
Ap - A(p+\dd p) = \rho g A \dd h \Rightarrow \dd p = -\rho g \dd h
\label{eq:atmforcebalance}
\end{equation}
%
Using the ideal gas law $p=\rho R T/M$, where $R=8.314$ J/(K$\cdot$mol) is the
universal gas constant and $M$ is the gas molecular mass ($0.029$ kg/mol for
air), we can substitute $\rho$ and integrate this equation to obtain an 
exponential evolution of pressure with height:
%
\begin{equation}
\frac{\dd p}{p} = - \frac{gM}{RT}  \dd h \Rightarrow 
p=p_0 \exp\left(\frac{-h}{h_0}\right)
\label{eq:exponentialatm}
\end{equation}
%
where $p_0$ is the pressure at $h=0$ and $h_0=RT/(gM)$.

%----------------------------------------------------------------------
\paragraph{Solar radiation}
%----------------------------------------------------------------------

The Sun emits a tremendous amount of electromagnetic radiation at all 
frequencies and in all directions. At the Earth, we receive about
%
\begin{equation}
S_\Earth = 1366 \mbox{ W/m}^2
\end{equation}
%
of radiation. Naturally, the power per unit area (i.e. the \emph{irradiance})
increases as we get closer to the Sun, and decreases as we get away from it.

Due to conservation of energy, the irradiance $S(r)$ integrated over a the 
surface of a sphere of radius $r$ from the Sun is a constant independent of 
$r$:
%
\begin{equation}
4\pi r^2 S(r) = 4\pi r_\Earth S_\Earth \Rightarrow 
S(r)=S_\Earth \frac{r_\Earth^2}{r^2}
\end{equation}
%
where $r_\Earth = 150$ million km (1 astronomical unit) is roughly the 
distance between the Earth and the Sun. This formula is useful to compute the 
solar irradiance anywhere in the solar system, if we know the distance to the 
Sun, $r$.

Solar irradiance affects how much electric power we can generate with solar 
arrays, and it defines how hot our spacecraft will get when illuminated.
It also affects how much solar radiation pressure our spacecraft will feel.




